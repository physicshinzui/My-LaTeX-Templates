\documentclass[a4paper,12pt]{report}

% パッケージ
\usepackage{amsmath}  % 数学表現
\usepackage{graphicx} % 画像挿入
\usepackage{hyperref} % ハイパーリンク
\usepackage{geometry} % レイアウト調整
\usepackage{fancyhdr} % ヘッダーとフッター

% ページレイアウト
\geometry{
    top=30mm,
    bottom=30mm,
    left=25mm,
    right=25mm
}

% ヘッダーとフッターの設定
\pagestyle{fancy}
\fancyhead[L]{}
\fancyhead[C]{\leftmark}
\fancyhead[R]{\thepage}
\fancyfoot[L]{}
\fancyfoot[C]{}
\fancyfoot[R]{}

% タイトル情報
\title{レポートタイトル}
\author{著者名}
\date{\today}

\begin{document}

% タイトルページ
\maketitle

% 目次
\tableofcontents
\newpage

% 章1
\chapter{はじめに}
ここではレポートの背景や目的について説明します。

% 章2
\chapter{理論背景}
ここでは理論的な背景や関連する研究について説明します。

% セクション
\section{数式例}
以下に数式の例を示します。
\begin{equation}
    E = mc^2
\end{equation}

% 図の挿入
\section{図の挿入}
図\ref{fig:example}に例を示します。
\begin{figure}[h]
    \centering
    \includegraphics[width=0.5\textwidth]{example-image}
    \caption{例の図}
    \label{fig:example}
\end{figure}

% 章3
\chapter{実験結果}
ここでは実験結果について記述します。

% 章4
\chapter{結論}
ここではレポートの結論と今後の展望について述べます。

% 参考文献
\chapter*{参考文献}
\addcontentsline{toc}{chapter}{参考文献}
\bibliographystyle{unsrt}
\bibliography{references}

\end{document}