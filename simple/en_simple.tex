\documentclass[12pt,dvipdfmx]{article}
%===font
%\usepackage{mathpazo}    % or \usepackage{mathptmx} for Times-like

% === authors' affilication ===
\usepackage{authblk} % 
% =====================

\usepackage{graphicx}
% === Citation ===
\usepackage{cite} % show not [1,2,3] but [1-3]

%===
\usepackage{geometry}
\geometry{
    top=10mm,
    bottom=30mm,
    left=25mm,
    right=25mm
}

\usepackage{xcolor}

\usepackage[normalem]{ulem}
%Usage: \sout{}, \uline{}, \uwave{}

%=== Revise ===
\newcommand{\Add}[1]{\textcolor{red}{#1}}     % 追加
\newcommand{\Erase}[1]{\textcolor{red}{\sout{#1}}} % 削除
\newcommand{\Comment}[1]{\textcolor{blue}{#1}} % コメント
\usepackage[colorinlistoftodos]{todonotes}
%============

%---Words in Definition, theorem, proof are written as non-italic.
\usepackage{amsthm,amsmath,amssymb}
\usepackage{bm}
\usepackage{cancel}  % in equation mode, use xcancel{}
\usepackage{physics}
\theoremstyle{definition} % non-italic body text in the following environment
\newtheorem{dfn}{Definition}
\newtheorem{thm}{Theorem}
\newtheorem{prop}{Proposition}
\newtheorem{lem}{Lemma}
\newtheorem{cor}{Corollary}
\newtheorem{ex}{Example}
\newtheorem{exe}{Exercise}
\newtheorem{property}{Property}
\newtheorem{rem}{Remark}
\newtheorem{fact}{Fact}
\newtheorem{res}{Result}
\newtheorem{req}{Requirement}

% ================================================================
\title{Simple Template}
\author[1]{Shinji Iida\thanks{shinji.iida.ac@gmail.com}}
\author[1,2]{Bob\thanks{xxx.com}}
\affil[1]{Department of Data Science, Kitasato University,Kanagawa, Japan}
\affil[2]{YYY Univerisity}
\date{\empty}
\setcounter{Maxaffil}{0}
\renewcommand\Affilfont{\itshape\small}
% ================================================================

\begin{document}
\maketitle
\begin{abstract}
This is the abstract of the paper. It provides a brief summary of the research, including the problem, methods, results, and conclusions.
\end{abstract}

\section*{Introduction}
Ikebe et al. conducted...\cite{Ikebe2016-sg}

\begin{dfn}[Cauchy sequence]
For every $\epsilon > 0$, there exists $n_0 \in \mathbb{N}$ such that for every $m, n \ge n_0 \ (m,n \in \mathbb{N}),  \lvert a_m  - a_n  \rvert < \epsilon $.
\end{dfn}	

\begin{thm}[Archimedes axiom]
For every $a, b \in \mathbb{R}$ and $a > 0$, there exists $N \in \mathbb{N}$ such that $Na > b$.
\end{thm}

\begin{align}
y=ax
\end{align}

\begin{figure}[htbp]
    \centering
    \framebox(120,100){} 
    \caption{Dummy figure}
    \label{fig:dummy}
\end{figure}

\subsection*{Acknowledgements}

%=== Reference === 
\bibliography{../paperpile.bib}

%\addcontentsline{toc}{chapter}{References}
\bibliographystyle{unsrt}
%\bibliography{references}

\end{document}
