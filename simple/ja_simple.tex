\documentclass[12pt,dvipdfmx]{jlreq} % 日本語対応

% === authors' affilication ===
\usepackage{authblk}  

% === 画像処理 ===
\usepackage{graphicx}

% === Citation(和文対応) ===
\usepackage[numbers]{natbib}  % 日本語文書では natbib を推奨

% === ページ余白設定(日本語向け調整) ===
\usepackage{geometry}
\geometry{
    tmargin=20mm,  % 上の余白
    bmargin=30mm,  % 下の余白
    lmargin=25mm,  % 左の余白
    rmargin=25mm   % 右の余白
}

% === 文字装飾(修正用) ===
\usepackage{xcolor}
\usepackage[normalem]{ulem}  % \sout{}, \uline{}, \uwave{}

% === 修正用マクロ ===
\newcommand{\Add}[1]{\textcolor{red}{#1}}     % 追加
\newcommand{\Erase}[1]{\textcolor{red}{\sout{#1}}} % 削除
\newcommand{\Comment}[1]{\textcolor{blue}{#1}} % コメント
\usepackage[colorinlistoftodos]{todonotes}

% === 数式環境 ===
\usepackage{amsthm,amsmath,amssymb}
\usepackage{bm}
\usepackage{cancel}  % 数式の取り消し線
\usepackage{physics}

% === 定理環境(和文表記) ===
\theoremstyle{definition}
\newtheorem{dfn}{定義}
\newtheorem{thm}{定理}
\newtheorem{prop}{命題}
\newtheorem{lem}{補題}
\newtheorem{cor}{系}
\newtheorem{ex}{例}
\newtheorem{exe}{演習}
\newtheorem{property}{性質}
\newtheorem{rem}{注意}
\newtheorem{fact}{事実}
\newtheorem{res}{結果}
\newtheorem{req}{要請}

% ================================================================
\title{日本語テンプレート}
\author[1]{アリス\thanks{xxx.com}}
\author[1,2]{ボブ\thanks{yyy.com}}
\affil[1]{XXX大学}
\affil[2]{YYY大学}
\date{\empty}
\setcounter{Maxaffil}{0}
\renewcommand\Affilfont{\itshape\small}
% ================================================================

\begin{document}
\maketitle

\begin{abstract}
この論文の概要を記述する。研究の課題、方法、結果、結論を簡潔に述べる。
\end{abstract}

\section*{序論}
池辺らは...\cite{Ikebe2016-sg}

\begin{dfn}[コーシー列]
任意の $\epsilon > 0$ に対し、ある $n_0 \in \mathbb{N}$ が存在して、
任意の $m, n \geq n_0 \ (m,n \in \mathbb{N})$ に対し、
\[
\lvert a_m  - a_n  \rvert < \epsilon
\]
が成り立つとき、数列 $\{a_n\}$ は **コーシー列** という。
\end{dfn}	

\begin{thm}[アルキメデスの公理]
任意の $a, b \in \mathbb{R}$ で $a > 0$ に対し、ある $N \in \mathbb{N}$ が存在して
\[
Na > b
\]
を満たす。
\end{thm}

\begin{align}
y=ax
\end{align}

\begin{figure}[htbp]
    \centering
    \framebox(120,100){} 
    \caption{ダミー図}
    \label{fig:dummy}
\end{figure}

\subsection*{謝辞}
本研究の遂行にあたり、XXX氏に感謝する。

%=== 参考文献 === 
\bibliographystyle{unsrt} % 文献の並び順
\bibliography{../paperpile.bib} % BibTeX 参照

\end{document}

